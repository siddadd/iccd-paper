
Due to the capacity of human vision systems for highly complex processing at very low power, there is growng interest in emulating such systems for video analytics applications.
Deriving inspiration from the energy-efficiency of the visual cortex, many brain-inspired algorithms and architectures have been proposed for this purpose~\cite{Nere2011,Chen2014,Kestur2012}. %[YiranChen UPitt, Qingriu Syracuse, NEC CNN]. 

%In order to achieve fast and energy efficient implementations of these algorithms, it has been found general-purpose processors are inadequate, even with highly tuned software optimizations. Consequently, a host of application-specific accelerators are employed, allied with a general purpose core for other tasks such as synchronization and co-ordination between the individual specialized computation engines. 


Most works in this domain have focused mainly on enhancing the performance and energy efficiency of the computational fabrics and do not address the inefficiencies of the main memory system. The memory system contributes between 10-30\% of the overall power of embedded video systems and mobile phones~\cite{CarrollAaronHeiser2010}. 
%For instance, our system uses accelerators for realizing tasks such as object saliency, object recognition and action recognition in a fast and energy-efficient manner, with low area overhead. Ho
The increasing memory size in new generations of embedded systems and the use of stacked 3D architectures that increase on-chip temperatures have drawn increasing attention on reducing the memory refresh energy. Consequently, there have been sustained efforts to introduce new power-efficient techniques such as Low Power Auto Self Refresh, Temperature Controlled Refresh, Refresh Pausing, Fine Granularity Refresh and Data Bus Inversion in new memory standards such as DDR4~\cite{jedec-sdram-standards}.  However, software exploitation of these advanced hardware features has generally lagged~\cite{Mukundan2013,refresh-pausing-taco2014}. 

Tuning DRAM refresh based on the data characteristics has been proposed as early as 1998~\cite{islped98}. Recently, a software approach, termed as \emph{Flikker} was proposed that relies on the user to annotate critical and non-critical parts. This technique has been used to effectively exploit a modified version of the standard Partial Array Self Refresh (PASR) hardware mechanism~\cite{Liu2011}. It also allows refresh rates to be different for critical and non-critical sections of the memory and conserve the refresh energy. 
%There have been several other works in the design of accelerator-based systems for vision analytics. For instance~\cite{Effex} proposes \emph{EFFEX}, a specialized heterogeneous multicore design for vision applications in the embedded domain. In this paper, the authors have proposed customizations to the memory hierarchy, such as hardware-software co-optimized patch memory architectures. However, the memory optimizations are mainly restricted to the software level and there remains substantial scope for optimizing the memory architecture itself, especially from the point of view of improving energy and bandwidth utilization.

In this work, we focus on the unique opportunities provided by real-time embedded video analytics applications for reducing the memory refresh energy. The analysis is based on an real-time image detection and recognition system that has been emulated on a FPGA based platform. This system detects objects of interest using an attention algorithm and then the object can be picked up from a subsequent frame where it is recognized to be of a particular class. The recognized object can be further passed to a next level for more detailed recognition. This system can be useful in a range of end applications such as unmanned air-vehicles, security cameras, visual-aids for visually-impaired and automatic weapon systems. We thus make the following contributions with regard to reducing the memory refresh energy based of this video analytics application.

\begin{itemize}[leftmargin=*]
\item We recognize that in streaming data, the lifetime of some parts of the data are significantly less than the refresh periods of a DRAM. Therefore, we completely disable refresh in these parts of the memory. 
%This is an enhancement to current techniques that reduce refresh times instead of completely shutting off the refresh in non-critical portions of the data. 
We are also able to eliminate refresh for portions of the memory that are guaranteed to be accessed within a specific time period due to the application specific nature of the embedded video system.
\item We automatically recognize portions of an image as critical based on the saliency-recognition algorithms employed in brain-inspired vision algorithms and selectively refresh portions of data.
 %This is in contrast to prior efforts that have focused on static methods such as manual annotations of critical and non-critical regions. 
%For example~\cite{Liu2011} annotates the code and certain data structures such as pointers to list of frames as critical while considering the image data itself as non-critical. This approach limits the granularity of such annotations, especially when such criticality is data-driven. 
In contrast, to~\cite{Liu2011} which statically annotates portions of the code and partitions them into critical and non-critical regions, our automated system can exploit both data dependent and task dependent information to identify salient regions within a single image frame. 
%The human visual cortex filters a significant amount of the raw visual stimuli for further processing by using attention mechanisms to identify the salient parts of the input. 
%The salient features are determined by a combination of the low-level features of the stimuli as well as the feedback from the visual task being performed by the person. 
Thus, only the salient regions of an image frame can be refreshed while allowing the rest of the image to degrade without need for refresh. 
\item We dynamically estimate the useful lifetime of buffered salient image data for further temporal analysis to predictively turn-off refresh for portions of the buffered data. When salient portions of the image are classified by the recognition engine based on the resulting class and the visual task to an accomplished by the vision system, the lifetime of the salient portion can be estimated. 
We also propose the underlying architecture to support such a visual pipeline.
%For example, if a salient region is determined to be a chair, instead of a luggage, and the task is to identify unattended luggage, that salient region can be marked as being less critical for the task and its refresh rate can be reduced (or refresh turned-off). Similarly, we can turn off the refresh for the buffered salient regions with luggage, if a person is identified next to that salient luggage later in the temporal sequence. 
%\item Our scheme yields savings of 88.1\% in refresh power and 14.8\% in total power, as compared to a standard DRAM refresh scheme.
\end{itemize}

The rest of this paper is organized as follows.
In Section~\ref{sec:background}, we provide an overview of the vision-based system and the accelerators that are employed in it. We also describe the memory hierarchy used in our system and the possible scope for improving the performance and power efficiency.
Section~\ref{sec:architecture} describes our proposed architecture design and highlights the additions over a baseline system.
Section~\ref{sec:results} enumerates our experimental results along with the performance and energy benefits that our design provides. Finally, we conclude with Section~\ref{sec:conclusion}.


